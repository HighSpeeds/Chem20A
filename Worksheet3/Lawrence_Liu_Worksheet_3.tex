\documentclass[12pt]{article}
\author{Lawrence Liu, UID: 405749034}
\usepackage{subcaption}
\usepackage{graphicx}
\usepackage{amsmath}
\usepackage{pdfpages}
\newcommand{\Laplace}{\mathscr{L}}
\setlength{\parskip}{\baselineskip}%
\setlength{\parindent}{0pt}%
\usepackage{xcolor}
\usepackage{listings}
\definecolor{backcolour}{rgb}{0.95,0.95,0.92}
\usepackage{amssymb}
\usepackage{empheq}

\newcommand*\widefbox[1]{\fbox{\hspace{2em}#1\hspace{2em}}}
\lstdefinestyle{mystyle}{
    backgroundcolor=\color{backcolour}}
\lstset{style=mystyle}
\title{Chem 20A Worksheet 3}
\begin{document}
\maketitle
\section*{Problem 1}
We have that
$$\lambda=\frac{h}{p}=\frac{h}{mv}$$
Thus we have that
$$v=\frac{m\lambda}=3 095 274 m/s$$
And thus we have that the electron kenetic energy is
$$E=\frac{1}{2}mv^2=\boxed{4.363\cdot 10^{-18}J}$$
\section*{Problem 2}
\subsection*{(a)}
We have from the heisenberg uncertainty principle that
$$\Delta x\Delta p\geq\frac{\hbar}{2}$$
Thus we have that
$$m_e\Delta x\Delta v\geq\frac{\hbar}{2}$$
Thus we have that
$$\Delta v\geq\frac{\hbar}{2m_e\Delta x}=\boxed{57 883m/s}$$
\subsection*{(b)}
We have that we can also rewrite the result from part (a) as 
$$\Delta v\geq\frac{h}{4\pi m_e\Delta x}$$
So in our case of $h=1js$ we would have that 
$$\Delta v\geq\frac{1}{4\pi m_e\Delta x}=8.735\cdot 10^{37}m/s$$
\section*{Problem 3}
\subsection*{(a)}
For a electron in a box we have that the energy is given by 
$$E_n=\frac{n^2\pi^2\hbar^2}{2m_eL^2}$$
Therefore we have that the ground state is
$$E_1=\frac{\pi^2\hbar^2}{2m_eL^2}=3.355 \cdot 10^{-18}J$$
And the first excited state is
$$E_2=\frac{4\pi^2\hbar^2}{2m_eL^2}=1.342 \cdot 10^{-17}J$$
Thus we have that the minimum energy required to excite the electron is
$$\Delta E=E_2-E_1=\boxed{1.007 \cdot 10^{-17}J}$$
\subsection*{(b)}
We have that the wavefunction for the ground state is 
$$\psi_1(x)=\sqrt{\frac{2}{L}}\cos(\frac{\pi x}{L})$$
Therefore the probability $P_1$ of finding the box inside the interval $[0.5,0.7]$ angstrom is given by
$$P_1=\int_{0.5}^{0.7}|\psi_1(x)|^2dx=\boxed{\frac{2}{1.34\AA}\int_{0.5\AA}^{0.7\AA}\cos^2\left(\frac{\pi x}{1.34\AA}\right)dx}$$
Likewise we have that the wavefunction for the first excited state is
$$\psi_2(x)=\sqrt{\frac{2}{L}}\sin(\frac{2\pi x}{L})$$
Therefore the probability $P_2$ of finding the box inside the interval $[0.5,0.7]$ angstrom is given by
$$P_2=\int_{0.5}^{0.7}|\psi_2(x)|^2dx=\boxed{\frac{2}{1.34\AA}\int_{0.5\AA}^{0.7\AA}\sin^2\left(\frac{2\pi x}{1.34\AA}\right)dx}$$
\end{document}